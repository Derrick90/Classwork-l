\documentclass[options]{article}

 \usepackage[
    top    = 2.75cm,
    bottom = 2.50cm,
    left   = 4.00cm,
    right  = 3.50cm]{geometry}

\usepackage[parfill]{parskip}

\title{Impact of Social Media to Youth}
\author{Wafula Derrick (216021715, 16/U/20275/EVE) \thanks{supervisor: Ernest Mwebaze}}
\date{%
    Makerere University\\%
    Feb 25, 2018
}


\begin{document}
\begin{titlepage}
\maketitle
\end{titlepage}





\section{\textbf{ Introduction}} 
Humans are naturally very social beings and the way we communicate is vital aspect of our lives. More importantly, communicating over long distances is something that people once struggled with, but thanks to recent advances in technology, it is much easier for us today. Where we once relied on smoke signals and even carrier pigeons, we have many more options now. During the 1800s, there was a rush among inventors to develop newer and better ways to allow long distance and mass communication. In the 1800’s there was an explosion in the ways we communicated globally. Telegraphs, radio and telephones made a dramatic difference in how information can be conveyed. The continuous search for the innovative ways of communication lead us to the development of different social media. In recent years we have seen another boom in communication. With new technology we are now able to communicate across the globe (and even into space) almost instantaneously.
Social media is the interaction among people in which they create share or exchange information and ideas in virtual communities and networks, Govender, et al(2013). Social media depend on mobile and web-based technologies to create highly interactive platforms through which individuals and communities share, co-create, discuss, and modify user-generated content. They introduce substantial and pervasive changes to communication between organizations, communities, and individuals, Megan A.Pumper et al(2011). The increased use of the Internet as a new tool in communication has changed the way people interact. Recently, a new means of online communication has emerged with its own set of idiosyncrasies. This new communication style occurs through the use of social networking sites.


\subsection{\textbf{Research Background}}
Kyriaki et al (2013) examines the relationship of Social Networking Site (SNS) problematic usage with personality characteristics and depressive symptomatology. A sample of 143 young adults in Greece varying from 18 to 34 years of age completed four questionnaires on personality characteristics, depressive symptomatology, problematic SNS usage and socio-demographic factors. Problematic SNS usage is significantly and positively related to depression and Neuroticism, while negatively associated with Agreeableness. However, problematic use of SNS is not related to Conscientiousness, Openness to Experience and Extraversion, although the latter was found to be negatively associated with depression. Collectively, personality variables, depression and daily average usage account for about 33% of the variance in predicting problematic SNS usage.
In contrast to previous research findings, age and gender are not found to be related to either problematic SNS usage or depressive symptomatology. However, place of residence is associated with Neuroticism and Problematic Use, with participants from rural areas exhibiting higher scores than participants from urban areas. Finally, Neuroticism and the average daily use of SNS have been proven to be good predictors of problematic SNS usage. Bahire et al (2014) compared real-life friendships with friendships formed through SNS, in fulfilling the attachment needs of students at the English Preparatory School at the Eastern Mediterranean University. It was designed as a case study of English Preparatory School students at the Eastern Mediterranean University during the 2011 academic year. Of the 600 total students enrolled in the course, 100 (n = 77 female, 23 male) were selected using random sampling to participate in the study. They found statistically significant differences between attachments formed in real life and those formed through SNS.
 
  \bigbreak
Therefore, they determined that SNS play a significant role in satisfying the need for attachment among young people who are at the outset of their tertiary education. P. Raghavendra et al (2013) investigated the effectiveness of tailored one-on-one support strategies designed to facilitate social participation of youth with disabilities through the use of the Internet for social networking. Methods Eighteen youth aged 10–18 years with cerebral palsy, physical disability or acquired brain injury received support, training and assistive technology at their home to learn to use the Internet for building social networks. The Canadian Occupational Performance Measure (COPM) and Goal Attainment Scale (GAS) were used to evaluate objective changes in performance and satisfaction. Interviews showed that youth were positive about the benefits of hands-on training at home leading to increased use of the Internet for social networking. The Internet could be a viable method to facilitate social participation for youth with disabilities.  \bigbreak



\subsection{\textbf{Problem Statement}}
The study was design to analyzed the impact of social media on youth, how social media is influencing on youth in different aspects of social life, political awareness, religious practices, educational learning, trends adopting, sports activities and so on.
\subsection{\textbf{Objectives}}
The study has taken overall objectives of identifying the factors of social networking sites and its impacts on youth and examining whether the social networking sites influence the lifestyle of youth.

\subsubsection{\textbf{Main Objective}} 
To identify impact of Social Networking Sites on youth in universities and workplaces in Uganda (more specifically urban centers).

\subsubsection{\textbf{Specific Objectives}}

\begin{itemize}
  \item To find out the usage of Social Networking sites among youth.
  \item To identify how Social Networking Sites benefitted to the youth.
  \item To know whether there is any negative effects on the personal life of youth by Social Networking Sites.
   \item To know the positive impacts created by Social Networking Sites among youth.
   \item To identify how Social Networking sites affect the lifestyle of youth.
\end{itemize}


\subsection{\textbf{Scope}}
This research is aimed at youths urban setting more specifically at  higher institutions of learning and those at work places.

\subsection{\textbf{Purpose of the Study}}
This study is expedient to apply social media in right direction for youth and create cognizance among youth that proper use of social media become a solid tool to educate, inform and groomed the mentality level of youth social media refine their living style of public especially for youth it is also create an responsiveness that how it is effecting the social life the deteriorate social norm, society standards and ethics of society and create awareness among youth the aspect of social media.

\section{\textbf{Research Scope}}
The study focuses more on the university level has received far more attention than the high school level. However, those studies gave some indication about the pervasiveness of a number of problems that relate to this study.

\section{\textbf{Methodology}}

\begin{itemize}
\item Research design:
In this study, design adopted is descriptive which includes surveys and fact- finding enquires of different kinds. The major purpose of descriptive research is to give a description of the state of affairs as it exists at present, because the researcher has no control over the variables and can only report as to what had happened or what is happening. It also attempts to discover the causes even when they cannot control the variables. The descriptive research design is considered as the ideal design to examine the impact of social networking sites on youth

\item  Source of data collection:
For the purpose of this study, it was decided to collect the data with the help of a questionnaire. Structured random sampling method of questioning was adopted. The questionnaire was prepared on the basis of collected information and reviews about the social networking sites, youth and their lifestyle.

\item Collection of data:
In this study, the researcher used questionnaire method as the tool. All the questions are structured on the basis of fulfilling the objectives of the study. A total of 250 respondents were chosen for the study in this area

\item Collection of data:
In this study, the researcher used questionnaire method as the tool. All the questions are structured on the basis of fulfilling the objectives of the study. A total of 250 respondents were chosen for the study in this area.

\item  Sampling:
The researcher interviewed the young people both male and female. Samples are selected from students of different colleges and higher secondary schools and youth working in various companies and offices in Kampala Uganda on the convenience of the researcher, because of the difficulties in meeting them and getting their co-operation.. The sample size of the present study is determined to be 250 young people.
\end{itemize}







\end{document}